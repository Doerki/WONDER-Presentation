%%%%%%%%%%%%%%%%%%%%%%%%%%%%%%%%%%%%%%%%%
% Beamer Presentation
% LaTeX Template
% Version 1.0 (10/11/12)
%
% This template has been downloaded from:
% http://www.LaTeXTemplates.com
%
% License:
% CC BY-NC-SA 3.0 (http://creativecommons.org/licenses/by-nc-sa/3.0/)
%
%%%%%%%%%%%%%%%%%%%%%%%%%%%%%%%%%%%%%%%%%

%----------------------------------------------------------------------------------------
%	PACKAGES AND THEMES
%----------------------------------------------------------------------------------------

\documentclass{beamer}

\mode<presentation> {

% The Beamer class comes with a number of default slide themes
% which change the colors and layouts of slides. Below this is a list
% of all the themes, uncomment each in turn to see what they look like.

%\usetheme{default}
%\usetheme{AnnArbor}
%\usetheme{Antibes}
%\usetheme{Bergen}
%\usetheme{Berkeley}
%\usetheme{Berlin}
%\usetheme{Boadilla}
%\usetheme{CambridgeUS}
%\usetheme{Copenhagen}
%\usetheme{Darmstadt}
%\usetheme{Dresden}
%\usetheme{Frankfurt}
%\usetheme{Goettingen} % fav 2 - nav right
%\usetheme{Hannover}
%\usetheme{Ilmenau}
%\usetheme{JuanLesPins}
%\usetheme{Luebeck} % fav1 - nav top
%\usetheme{Madrid}
%\usetheme{Malmoe}
%\usetheme{Marburg}
\usetheme{Montpellier} % fav2 - nav top
%\usetheme{PaloAlto}
%\usetheme{Pittsburgh}
%\usetheme{Rochester}
%\usetheme{Singapore}
%\usetheme{Szeged}
%\usetheme{Warsaw}

% As well as themes, the Beamer class has a number of color themes
% for any slide theme. Uncomment each of these in turn to see how it
% changes the colors of your current slide theme.

%\usecolortheme{albatross}
%\usecolortheme{beaver}
%\usecolortheme{beetle}
%\usecolortheme{crane}
%\usecolortheme{dolphin}
%\usecolortheme{dove}
%\usecolortheme{fly}
%\usecolortheme{lily}
\usecolortheme{orchid} %FAV1
%\usecolortheme{rose}
%\usecolortheme{seagull}
%\usecolortheme{seahorse}
%\usecolortheme{whale}
%\usecolortheme{wolverine}

%\setbeamertemplate{footline} % To remove the footer line in all slides uncomment this line
%\setbeamertemplate{footline}[page number] % To replace the footer line in all slides with a simple slide count uncomment this line

%\setbeamertemplate{navigation symbols}{} % To remove the navigation symbols from the bottom of all slides uncomment this line
}

\usepackage{graphicx} % Allows including images
\usepackage{booktabs} % Allows the use of \toprule, \midrule and \bottomrule in tables

%Verwendete Codierung
\usepackage[utf8]{inputenc}
%Zur Einbindung von Grafiken
\usepackage{graphicx}
\usepackage{color}
% Dieses Paket bietet LaTeX die Möglichkeit Text um ein Bild anzuordnen.
\usepackage{wrapfig}

%Zum einfügen von Sonderzeichn und Währungen mittels \textxxxx befhels
\usepackage{textcomp}

% Paket erzeugt ein anklickbares Verzeichnis in der PDF-Datei.
\usepackage{hyperref}

% Paket für 1,5-zeiligen Zeilenabstand.
% \usepackage{setspace}

% Mit dem Multicol-Paket können Textteile nebeneinander arrangiert werden.
\usepackage{multicol}

% Das Acronym-Paket wird für das Abkürzungsverzeichnis benötigt.
\usepackage{acronym}

%Ermöglicht die automatische Trennung von Worten mit Umlauten
\usepackage[T1]{fontenc}

 % um u. a. Zeilenumbrüche bei zu langen URLs im Literaturverzeichnis zu
 %erzwingen
%\usepackage[colorlinks=false, pdfborder={0 0 0}]{hyperref}

%Einbindung der Sprachpakete zur Rechtschreibkorrektur und Silbentrennung
\usepackage[english, ngerman]{babel}

%\usepackage{listings}
%--------------------------------------------------------------------
% Pakete
%--------------------------------------------------------------------

% KOMA-Script

\usepackage[table]{xcolor}          % Erweitertes Farbpaket mit vielen 
%Farbmodellen (notwenidig zur Farbdefinition)

\usepackage{listings}               % Quellcode einbinden und formatieren
\renewcommand{\lstlistlistingname}{Verzeichnis der Listings}
\definecolor{darkgreen}{rgb}{0,.4,0}
\definecolor{darkviolett}{rgb}{.4,0,.4}
\newcommand{\listingswidth}{\textwidth-1cm}
\lstset{
    %language=C,                  % oder  C++, Pascal, {[77]Fortran}, ...
    numbers=left,                   % Position der Zeilennummerierung
    firstnumber=auto,               % Erste  Zeilennummer
    basicstyle=\ttfamily\small,     % Textgröße  des Standardtexts
    keywordstyle=\ttfamily\color{darkviolett},    % Formattierung Schlüsselwörter
    commentstyle=\ttfamily\color{darkgreen},       % Formattierung Kommentar
    stringstyle=\ttfamily\color{blue},             % Formattierung Strings
    numberstyle=\tiny,              % Textgröße der Zeilennummern
    stepnumber=1,                   % Angezeigte Zeilennummern
    numbersep=5pt,                  % Abstand zw. Zeilennummern und Code
    aboveskip=0pt,                 % Abstand oberhalb des Codes
    belowskip=0pt,                 % Abstand unterhalb des Codes
    captionpos=b,                   % Position der Überschrift
    linewidth=8.3cm,               % TODO \textwidth-2em
    xleftmargin=10pt,               % Linke Einrückung
    frame=single,                   % Rahmentyp
       breaklines=true,             %% Zeilen umbrechen 
        prebreak={\rotatebox[origin=c]{270}{$\curvearrowright$}},  %% Vor 
        %%Zeilenumbruch Zeichen 
        %%setzen 
       breakautoindent=true         %% umbrochene Zeilen einrücken 
    showstringspaces=false          % Spezielles Zeichen für Leerzeichen
}

%----------------------------------------------------------------------------------------
%	TITLE PAGE
%----------------------------------------------------------------------------------------

\title[WONDER GUI]{WONDER GUI} % The short title appears at the bottom
%of every slide, the full title is only on the title page

\author{Danny Koppenhagen} % Your name
\institute[UCLA] % Your institution as it will appear on the bottom of every slide, may be shorthand to save space
{
Hochschule für Telekommunikation Leipzig \\ % Your institution for the title
%page
\medskip
\textit{Danny.Koppenhagen@hft-leipzig.de} %
%Your email address
}
\date{\today} % Date, can be changed to a custom date

\begin{document}
	\begin{frame}
		\titlepage % Print the title page as the first slide
	\end{frame}

	\begin{frame}
		\frametitle{Übersicht} % Table of contents slide, comment this block
		%out to
		%remove it
		\tableofcontents % Throughout your presentation, if you choose to use
		%\section{} and \subsection{} commands, these will automatically be
		%printed
		%on this slide as an overview of your presentation
	\end{frame}

\section{Anforderungen}
	\begin{frame}
		\frametitle{Anforderungen}
		\begin{block}{Aufgabe}
		Erstellung eines Frontends basierend auf dem WONDER API als eine
		Alternative
		für das bereits existierende Frontend.
		\end{block}
		\pause
		GUI Elemente und Funktionalitäten für:
		\begin{itemize}[<+->]
			\item Login/Logout  an den Domains (NodeJS/IMS)
			\color[rgb]{0,0.8,0}\checkmark
			\item Call Funktionen \color[rgb]{0,0.8,0}\checkmark
			\begin{itemize}[<2->]
				\item Call zu einer beliebigen URL
				\item Annahme eines eingehenden Calls
			\end{itemize}
			\item Chat \color[rgb]{0,0.8,0}\checkmark
			\item File-Transfer {\color[rgb]{0.8,0,0} X}
			\item Kontaktliste
			\begin{itemize}[<6->]
				\item Add {\color[rgb]{0,0.8,0}\checkmark},
				Modify {\color[rgb]{0.8,0,0} X}, Remove {\color[rgb]{0.8,0,0} X}
				\item Direktes Wählen aus der Kontaktliste
				{\color[rgb]{0,0.8,0}\checkmark}
				\item Presence der Kontakte (online/offline)
				{\color[rgb]{0.8,0,0} X}
			\end{itemize}
		\end{itemize}
	\end{frame}

\section{Umsetzung}
\subsection{Verwendete Frameworks und Libraries}
	\begin{frame}
		\frametitle{Verwendete Frameworks und Libraries}
		\begin{itemize}[<+->]
			\item jQuery\cite{p1}
			\begin{itemize}[<1->]
				\item Javascript Bibliothek
			\end{itemize}
			\item Bootstrap\cite{p2}
			\begin{itemize}[<2->]
				\item weit verbreitetes HTML/CSS/Javascript Framework
				\item optimiert für Responsive Layouts
			\end{itemize}
			\item Summernote\cite{p3}
			\begin{itemize}[<3->]
				\item grafischer Editor für Textfelder
			\end{itemize}
			\item Bootstrap video player\cite{p4}
			\begin{itemize}[<3->]
				\item HTML5 Video Player Plugin für Bootstrap/jQuery
				\item Optische Anpassung der HTML5 A/V Elemente
			\end{itemize}
		\end{itemize}
	\end{frame}

\subsection{Modulares Konzept}
	\begin{frame}
		\frametitle{Modulares Konzept}
		
		\only<1>{
			\begin{block}{Modulares Konzept}
			Ziel: GUI Elemente können mit Modulen ja nach Status ein- und
			ausgeblendet werden.
			\end{block}
		}
		\pause
		\only<2>{
			\begin{exampleblock}{vorher (main.js)}
			\lstinputlisting[language=html,
			linewidth=\textwidth,frame=none]{src/oldUI.js}
			\end{exampleblock}

		\pause

			\begin{exampleblock}{nachher (main.js)}
			\lstinputlisting[language=html,
			linewidth=\textwidth,frame=none]{src/newUI.js}
			\end{exampleblock}
		}
		\pause
		\begin{itemize}[<+->]
			\item Module
			\begin{itemize}[<1->]
				\item Start
				\item Login
				\item Audio
				\item Video
				\item Chat
			\end{itemize}
			\item Auslagerung in extra Datei
		\end{itemize}
	\end{frame}

\subsection{Kontaktmanagement}
	\begin{frame}[fragile]
		\frametitle{Kontaktmanagement}
		\begin{itemize}
			\item zunächst Speicherung im Local Storage
			\item Erzeugung eindeutiger Kontakt-IDs
		\end{itemize}
		%change default settings for syntax Highlighting
		\pause
		\begin{exampleblock}{Kontaktformat}
		\lstinputlisting[language=html,
		linewidth=\textwidth,frame=none]{src/contactJSON.js}
		\end{exampleblock}
	\end{frame}

\section{Vorführung}
		\begin{frame}
			\frametitle{Vorführung}
			\href{run:/Users/dannyk/Desktop/test.txt}
				{\beamerbutton{Öffnen}}
		\end{frame}

\section{Ausblick}
	\begin{frame}
		\frametitle{Ausblick}
		\begin{itemize}[<+->]
			\item Kontaktmanagement mit Datenbank verknüpfen
			\item Presence Anzeige
			\begin{itemize}[<2->]
				\item GUI bereits vorbereitet
			\end{itemize}
			\item File-Transfer Integration
		\end{itemize}
	\end{frame}

%------------------------------------------------
\begin{frame}[fragile]
\frametitle{References}
\tiny{
\begin{thebibliography}{99}
\bibitem{p1} jquery.com/ (Letzter Aufruf: 09.02.2014)
	\newblock Javascript Bibliothek
	\newblock \url{http://jquery.com/}
\bibitem{p2} getbootstrap.com (Letzter Aufruf: 09.02.2014)
	\newblock HTML/CSS/Javascript Framework
	\newblock \url{http://getbootstrap.com/}
\bibitem{p3} summernote.org (Letzter Aufruf: 09.02.2014)
	\newblock WYSIWYG Editor
	\newblock \url{http://summernote.org/}
\bibitem{p4} html5-ninja.com (Letzter Aufruf: 09.02.2014)
	\newblock Bootstrap video player
	\newblock
	\url{http://html5-ninja.com/#/item/Bootstrap-video-player-jQuery-plugin/5}

\end{thebibliography}
}
\end{frame}
%------------------------------------------------
\begin{frame}
\Large{\centerline{Vielen Dank}}
\Large{\centerline{für ihre Aufmerksamkeit!}}
\Large{\centerline{}}
\Large{\centerline{Fragen?}}
\end{frame}
%----------------------------------------------------------------------------------------

\end{document}
